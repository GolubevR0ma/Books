\documentclass[a6paper,14pt]{extarticle}

%\usepackage{amssymb,amsfonts,amsmath,mathtext,cite,enumerate,float}
\usepackage[T2A]{fontenc}
% шрифт как в ссср
% заработал после http://kalina.lug.ru/wiki/%D0%A3%D1%81%D1%82%D0%B0%D0%BD%D0%BE%D0%B2%D0%BA%D0%B0_%D0%BA%D0%BE%D0%BC%D0%BF%D0%BB%D0%B5%D0%BA%D1%82%D0%B0_%D1%88%D1%80%D0%B8%D1%84%D1%82%D0%BE%D0%B2_PSCyr_%D0%BD%D0%B0_TeXLive_%D0%B2_Ubuntu_9.04
\usepackage{pscyr}
\usepackage[utf8]{inputenc}
\usepackage[english=nohyphenation,russian=nohyphenation]{hyphsubst}
\usepackage[english,russian]{babel}
% Для поиска по русским буквам
\usepackage{cmap}
\usepackage{indentfirst}        % Отступ в первом абзаце.

\usepackage{geometry} %способ ручной установки полей
\geometry{top=0.5cm} %поле сверху
\geometry{bottom=1.25cm} %поле снизу
\geometry{left=0.2cm} %поле справа
\geometry{right=0.2cm} %поле слева

\linespread{0.95}

\usepackage{wrapfig}


\ifx\pdfoutput\undefined
\usepackage{graphicx}
\else
\usepackage[pdftex]{graphicx}
\fi

\usepackage{textcomp}


\renewcommand{\bfdefault}{b}
%\righthyphenmin=4 % Минимальное число символов при переносе - 2.
%\righthyphenmin=30 % Минимальное число символов при переносе - 2.


% LaTeX Paragraph Indenting
%\setlength{\parindent}{0pt}
\setlength{\parskip}{1ex}

\usepackage{parallel}

% позволяет задавать цвет текста и фона,
% как отдельного блока, так и всего
% документа
\usepackage[usenames,table]{xcolor}

% --------------------------------------
\newenvironment{answer}{%
   \bfseries
   \color{blue}
}{}
% --------------------------------------
%\setlength{\parindent}{0pt}

%\maketitle
% Нумерация страниц выключена
\pagestyle{empty}

\sloppy

% Буквица
\usepackage{type1cm}
\usepackage{lettrine}

% Создание гиперссылок. Пакет загружен последним,
% так как он переопределяет многие команды LaTeX.
\usepackage{hyperref}

% vim: fdm=syntax
